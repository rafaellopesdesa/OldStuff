\documentclass{article}

\usepackage{amsmath}

\title{Revis\~ao R\'apida de Efeito Fotoel\'etrico}
\author{Rafael Lopes de S\'a}
\date{\today}

\begin{document}
\maketitle

\section{Resumo}

O efeito fotoel\'etrico consiste num el\'etron de um metal absorvendo um f\'oton de forma que esse el\'etron se torna livre:

\begin{equation}
\text{el\'etron ligado} + \text{f\'oton} \rightarrow \text{el\'etron livre}
\end{equation}



Se a energia do f\'oton absorvida pelo el\'etron for maior que a energia de liga\c c\~ao do el\'etron com a estrutura met\'alica, o el\'etron vai se tornar livre e pode formar uma corrente el\'etrica. Quase todo problema sobre efeito fotoel\'etrico se resolve usando conserva\c c\~ao de energia. Conceitualmente:

\begin{equation}
\begin{split}
(\text{energia do f\'oton}) &- (\text{energia gasta para liberar o f\'oton da estrutura met\'alica}) = \\
&(\text{energia cin\'etica do el\'etron livre}) + (\text{energia potencial do el\'etron livre})
\end{split}
\end{equation}

A energia gasta para liberar o f\'oton da estrutura met\'alica \'e algo complicado de se calcular e, em geral, representamos apenas por um s\'imbolo $\phi$ e pelo nome ``fun\c c\~ao trabalho''. \'E uma propriedade do metal e n\~ao do f\'oton.

A energia de um f\'oton \'e proporcional \`a sua freq\"u\^encia:
\begin{equation}
E_f = hf,
\end{equation}
onde $h$ \'e chamado de \textbf{constante de Planck}. A energia cin\'etica \'e dada pela f\'ormula familiar:
\begin{equation}
E_c = \frac{1}{2}mv^2.
\end{equation}
J\'a a energia potencial depende do seu sistema. Usualmente uma bateria pode ser conectada \`a celula fotoel\'etrica criando uma diferen\c ca de potencial. O el\'etron tem que ent\~ao ir de contra (se o polo positivo da bateria estiver ligado ao cotodo) ou a favor (de o polo negativo da bateria estiver ligado ao catodo) esse potencial e gastar\'a ou, respectivamente, receber\'a uma energia dada por:
\begin{equation}
E_p = Q_e\times V,
\end{equation}
onde $Q_e$ \'e a carga do el\'etron e V \'e a diferen\c ca de potencial da bateria. Colocando todos os conceitos juntos:
\begin{equation}
hf - \phi = E_c + E_p = \frac{1}{2}mv^2 + Q_eV.
\end{equation}

Algumas coisas a se lembrar:
\begin{itemize}
\item No efeito fotoel\'etrico usual, cada f\'oton \'e absrovido por um el\'etron. Isso quer dizer que se a energia do f\'oton n\~ao for pelo menos a fun\c c\~ao trabalho $\phi$, n\~ao haver\'a corrente el\'etrica. No caso em que h\'a uma bateria tamb\'em, a energia do f\'oton tem que ser, pelo menos, a fun\c c\~ao trabalho mais a energia potencial provida pela bateria.
\item Se o el\'etron absorver um f\'oton de energia maior (isto \'e, de maior frequ\^encia), ele sair\'a com maior energia. Mas \textbf{n\~ao quer dizer que mais el\'etrons ser\~ao emitidos}.
\item Para emitir mais el\'etron, voc\^e precisa de mais f\'otons. Isso quer dizer uma luz incidente mais intensa.
\end{itemize}

\section{Constantes e unidades}

A unidade de energia no Sistema Internacional de unidades \'e o Joule. $1\,\text{J}$ \'e uma quantidade muito grande para efeitos at\^omicos e subat\^omicos. Uma unidade conveniente \'e o $\text{eV}$. 1 $\text{eV}$ \'e definido como a energia que 1 (um) el\'etron tem num potencial de 1 (um) Volt. Para converter para o SI, basta usar a carga do el\'etron:

\begin{itemize}
\item $1\,\text{eV} = 1.6\times 10^{-19}\text{J}$,
\item $1\,\text{J} = 1/(1.6\times 10^{-19}) \text{eV} = 6.24\times 10^{18}\,\text{eV}$ ,
\end{itemize}
pela pr\'opria defini\c c\~ao de $\text{eV}$ a carga el\'etrica fundamental \'e escrita como $e = 1.6\times 10^{-19}\,\text{C} = 1\,\text{eV/V}$.

Algumas constantes:

\begin{itemize}
\item $h = 6.626\times 10^{-34}\,\text{Js} = 4.136\times 10^{-15}\,\text{eVs}$,
\item $c = 3\times 10^{8}\,\text{m/s}$.
\item $hc = 1240\,\text{eV nm}$
\end{itemize}
O valor de $hc$ \'e conveniente porque, muitas vezes, \'e dado o comprimento de onda ($\lambda$) do f\'oton em vez da freq\"u\^encia. Essas duas quantidades se relacionam por:

\begin{equation}
c = \lambda f,
\end{equation}
logo, a energia de um f\'oton com comprimento de onda $\lambda$ \'e dada por:
\begin{equation}
E = hf = \frac{hc}{\lambda}.
\end{equation}

Algumas vezes tamb\'em \'e conveniente usar $\text{eV/c}^2$ como unidade de massa e $eV/c$ como unidade de momento linear. Nessa unidade, a massa do el\'etron é dada por:
\begin{equation}
m_e = 511 \text{keV/c}^2.
\end{equation}

\section{Alguns exemplos}


\end{document}
