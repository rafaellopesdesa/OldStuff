\documentclass{article}

\usepackage{amsmath, graphicx, amssymb, epigraph}

\title{Revis\~ao R\'apida de Processos At\^omicos e Nucleares}
\author{Rafael Lopes de S\'a}
\date{\today}

\begin{document}

\textbf{Quest\~ao 1a}

\begin{equation}
\begin{split}
&E_e = hf - \phi = (4.14\times 10^{-15}\,\text{eV s})\times (7.31\times 10^{14}\,\text{Hz})-2.28\,\text{eV} = (3.03 - 2.28)\,\text{eV}\\
& \Rightarrow \boxed{E_e= 0.75\,\text{eV}}
\end{split}
\end{equation}

\textbf{Quest\~ao 1b}

\begin{equation}
hf_{\text{min}} - \phi = 0 \Rightarrow f_{\text{min}} = \frac{\phi}{h} = \frac{2.28\,\text{eV}}{4.14\times 10^{-15}\,\text{eV s}} \Rightarrow \boxed{f_{\text{min}} = 0.55\times 10^{15}\,\text{Hz}}
\end{equation}

\textbf{Quest\~ao 1c}

\begin{equation}
\lambda_{\text{max}} = \frac{c}{f_{\text{min}}} = \frac{3\times 10^{8}\,\text{m/s}}{0.55\times 10^{15}\,\text{Hz}} \Rightarrow \boxed{\lambda_{\text{max}} = 545\,\text{nm}}
\end{equation}

\textbf{Quest\~ao 1d}

\vspace{0.3cm}
\framebox{N\~ao h\'a foto-el\'etrons pois $\lambda > \lambda_{\text{max}}$ (veja resposta da (1c)).}
\vspace{0.3cm}

\textbf{Quest\~ao 2}

\begin{equation}
\begin{split}
&E_{\gamma} + E_e(n=3) = E_e(n=9)\\
&E_{\gamma} + \left(-\frac{\text{Ry}}{3^2}\right) = \left(-\frac{\text{Ry}}{9^2}\right)\\
&E_{\gamma} = \text{Ry}\times\left(\frac{1}{9} - \frac{1}{81}\right) = 13.6\text{eV}\times\frac{8}{81}\\
&\Rightarrow \boxed{E_{\gamma} = 1.34\,\text{eV}}
\end{split}
\end{equation}

\textbf{Quest\~ao 3}
\vspace{0.3cm}

Chamando a massa de $m$, o efeito biol\'ogico relativo de $\text{RBE}$ (ie, quando maior o $\text{RBE}$ maior o dano biol\'ogico) e de $d$ a penetra\c c\~ao, temos:

\begin{equation}
\boxed{m_{\alpha} > m_{\beta} > m_{\gamma}}
\end{equation}

\begin{equation}
\boxed{\text{RBE}_{\alpha} > \text{RBE}_{\beta} \approx \text{RBE}_{\gamma}}
\end{equation}

\begin{equation}
\boxed{d_{\gamma} > d_{\beta} > d_{\alpha}}
\end{equation}

\textbf{Quest\~ao 4a}

\begin{equation}
\lambda = \frac{\ln 2}{\tau_{1/2}} = \frac{\ln 2}{2.7\,\text{dias}} \Rightarrow \boxed{\lambda = 0.257\,\text{dias}^{-1}}
\end{equation}

\textbf{Quest\~ao 4b}

\begin{equation}
N(0) = \frac{10^{-6}\,\text{g}}{198\,\text{g/mol}}\times 6\times 10^{23}\,\text{n\'ucleos/mol}\Rightarrow N(0) = 3.03\times 10^{15}\,\text{n\'ucleos}
\end{equation}

\begin{equation}
\boxed{\lambda N(0) = 0.257\,\text{dias}^{-1}\times 3.03 \times 10^{15}\,\text{n\'ucleos} = 0.779\times 10^{15}\,\text{n\'ucleos/dia}}
\end{equation}

\textbf{Quest\~ao 4c}

\begin{equation}
N(7) = N(0)\times e^{-\lambda\times 7}
\end{equation}

\begin{equation}
\begin{split}
&\lambda N(7) = \lambda N(0) \times e^{-\lambda\times 7\,\text{dias}} = 0.779\times 10^{15}\,\text{n\'ucleos/dia}\times e^{-0.257\times 7}\\
&\boxed{\lambda N(7) = 0.129\times 10^{15}\,\text{n\'ucleos/dia}}
\end{split}
\end{equation}

\textbf{Quest\~ao 4d}

\begin{equation}
{}^{198}_{79}\text{Au} \rightarrow {}^{198}_{80}\text{Hg} + {}^{0}_{-1}\text{X}
\end{equation}
\framebox{Logo, ${}^{0}_{-1}\text{X}$ \'e um decaimento $\beta-$ e as part\'iculas emitidas s\~ao um el\'etron e um anti-neutrino do el\'etron.}

\end{document}

